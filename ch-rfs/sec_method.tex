% !TEX root = ../main.tex
% \section{Modeling desiderata}
% \label{sec:desiderata}
% Our goal is to build a recommendation model that provably maximizes an
% evaluation metric. The model should also approximate other models that recommend
% items based on sets of attributes. We propose criteria for models that rank from
% sets, and then exhibit a model satisfying these desiderata:
% \begin{itemize}
% \item Order-invariance: if the input to the model includes an unordered set of
%   attributes, the output of the model should not depend on the order that the
%   set elements are fed to the model.
% \item Improved recall: training the model should provably improve the evaluation
%   metric of recall. Models that satisfy this desideratum make good recommenders.
%   (In contrast, it is unclear whether a model that reconstructs the training
%   data well will lead to maximum recall performance.)
% \item Universal approximation: the model should be able to approximate any other
%   order-invariant model. This reduces the need to compare to other models in
%   lieu of comparing parameterizations of a single model.
% \item Parameter-sharing: the model should share parameters across items with
%   similar sets of attributes. This enables the model to scale to large
%   real-world datasets.
% \end{itemize}
% The criterion of order-invariance is necessary due to the set-valued features
% associated with every item. The parameter-sharing desideratum further narrows
% the set of models and guides the development of architectures parameterizing a
% model. The desiderata of improved recall and universal approximation, if
% satisfied, frees practitioners to focus on building architectures for a single
% model. Intra-model comparisons, rather than arduous inter-model comparisons,
% enable the development of generic architectures that work across many datasets
% for ranking from sets.
\section{RankFromSets}
\label{sec:rankfromsets}
\acrfull{rfs} is a class of recommendation models that recommend items with attributes to users. Let~$u~\in~\{1, \ldots, N\}$ be a user, $m \in \{1, \ldots, M\}$ be an item, and $\yum~\in~\{0,1\}$ be a binary indicator where $1$ indicates user $u$ consumed item $m$. For each item $m$, there is an associated set of attributes $x_m \in \{0,1\}^{|V|}$ from a vocabulary of $V$ attributes. The observed data is a collection of user-item interactions $\{(u, m)\}$ and the sets of attributes associated with items $\{x_m\}$.

We assume that a recommendation model is given a budget of $K$ recommendations to be made for each user. In response, the recommender system produces a list of $K$ distinct recommendations $\mathbf{r}_u~=~(r_{u1}, \ldots, r_{uK})$ for each user. The goal of the recommendation task in this paper is to maximize the expected Recall@$K$,
\begin{equation}
\textrm{Recall}@K = \mathbb{E}_{u}\left[\frac{\sum_{r \in \mathbf{r}_u} y_{ur}}{\sum_{m} \yum}\right]\, ,
\label{eq:recall}
\end{equation}
with the expectation over users in the empirical distribution $\cD$.

We combine three techniques to maximize Recall@$K$ with \gls{rfs}. First, we cast recommendation as a classification task. Second, we learn user- and attribute-level embeddings. Statistical strength is shared between items with similar attributes by representing items as the mean of their attribute embeddings. Third, we scale \gls{rfs} to large datasets using a stochastic optimization-based negative sampling training procedure.

\gls{rfs} casts the recommendation problem as a classification task. Given a user-item pair $(u,m)$ and regression function $f$, \gls{rfs} learns to predict the probability that item~$m$~will be consumed by user $u$:
$$p(\yum = 1 \mid u, m) = \sigma\left( f \left (u, x_m\right) \right)\, ,$$
where $x_m$ is the set of attributes of item $m$ and $\sigma$ is the sigmoid function. Recommendations made by \gls{rfs} are the maximum likelihood set formed by ranking a set of items for a user according to the model $f(u, x_m)$. We motivate treating recommendation as classification with the following observation.
% \begin{equation}
% \label{eqn:argmax}
% \mathbf{r}_u(K) = \underset{\mathbf{r} \in \mathbb{N}^K}{\text{argmax}} \sum_{m\in\mathbf{r}} f \left( u, x_{m} \right) \, .
% \end{equation}
% Setup the problem here as a generic classification problem. the recommendations are going to be the argmax over the probabilities output by a classifier.
% \vspace{0.5cm}
\begin{prop}
\label{prop:maximizing-recall}
Let $u \in \mathcal{U}$ be a user, $m \in \mathcal{M}$ be an item, and
$y(u,m) \in \{0,1\}$ be an indicator of whether user $u$ logged item $m$. Let
$\mathcal{E}$ be the worst-case error for binary classifier $\hat{y}(u,m)$ on
any~$(u,m)$ pair drawn from the data $\mathcal{D}$,
\begin{equation*}
  \mathcal{E} = \max_{(u, m) \in \mathcal{D}} \mathbb{1}\left[ \hat y(u, m) \neq y(u, m) \right] \, .
\end{equation*}
A binary classifier with zero worst-case error ($\mathcal{E}=0$) maximizes
recommendation recall.
\end{prop}
\begin{proof}
  A model with zero worst-case error is a perfect classifier, assigning greater probability to data with positive labels than to data with negative labels. In other words, it ranks positive examples above negative examples. Recall@$K$ is measured by the fraction of items with positive labels in a ranking returned by the model. In a classifier that achieves zero worst-case error, positively-labeled datapoints must be ranked higher than other datapoints, maximizing recall.
\end{proof}

\Cref{prop:maximizing-recall} is simple, but conceptually important. Under the assumption that a perfect classifier exists, a consistent method for learning a classifier will be a consistent method for learning a recommendation system that targets expected recall. Put another way, recall is inherently binary: a model does or does not recall an item; an item is or is not in the top $K$ recommendations in the numerator of \Cref{eq:recall}. So the best one can hope to do if recall is used to assess recommendation performance is to train a binary classifier. In practice, as with any regression method, a perfect classifier is unachievable. \Cref{prop:maximizing-recall} is a guiding principle rather than a finite-sample guarantee of maximal performance. As we show in \Cref{sec:rfs-experiments}, the classification approach of \gls{rfs} performs well in practice.

For recommending items with attributes, \Cref{prop:maximizing-recall} says that building a classifier such as \gls{rfs} is optimal if we measure recommendation performance with recall. To parameterize the \gls{rfs} classifier, a regression function $f(u, x_m)$ is needed. A straightforward parameterization is an inner product,
\begin{equation}
\label{eqn:rankfromsets}
  f\left(u, x_m\right) = \theta_u^\top\left(\frac{1}{|x_m|}\sum_{j\in x_m}
  \beta_j + g(x_m)\right) + h(x_m) \, .
\end{equation}
Each element in the inner product regression function in \Cref{eqn:rankfromsets} has an intuitive interpretation. The user embedding $\theta_u \in \mathbb{R}^d$ captures the latent preferences for user $u$. This captures the individual-level tastes of a user and is analogous to the user preference vector in classical collaborative filtering or the row embedding in matrix factorization. The attribute embedding $\beta_j \in \mathbb{R}^d$ is the latent quality conveyed through item $m$ having attribute $j$. (The set $x_m$ contains only attributes with~$x_{mj}=1$. Attributes that are not associated with item $m$ are ignored.) The item embedding function~${g(x_m)\in\mathbb{R}^d}$~represents qualities not conveyed through the set of item attributes. This term in the regression function enables collaborative filtering by capturing unobserved patterns in item consumption such as popularity. We describe how to construct this function below. The scalar item intercept function $h(x_m) \in \mathbb{R}$ makes an item more or less likely due to availability.
% !TEX root = ../../main.tex
\begin{table*}[htb!]
\centering
\resizebox{1.0\textwidth}{!}{%
\begin{tabular}{p{80mm}|p{80mm}}
  \toprule
  Query Item & Nearest Item by Cosine Similarity \\
  \midrule
  Two scoops of Raisin Bran cereal, organic Moroccan green tea, almond milk,
  light honey, tap water, large banana, large strawberries
       &
         Vita Bee bread, salted butter, fresh medium tomatoes, large fried whole
         egg, small banana\\\hline
  Iceberg lettuce, cantaloupe cubes, diced honeydew melon, cherry tomatoes, olives, dry-cooked unsalted
  hulled sunflower seed kernels, chopped hard-boiled
  egg, cucumbers, dried cranberries, fat-free ranch dressing
       &
         Green leaf lettuce, chopped sweet red bell peppers, crumbled feta cheese,
         large hard-boiled egg,
         chopped cucumber, oil-roasted salted sunflower seeds, sliced radishes, sliced strawberries, pitted Calamata olives, fat-free
         balsamic vinegar\\\hline
  Boston roast pork, mackerel, artichoke hearts, spinach, pimiento-stuffed
  Manzanilla olives, carrots, mushrooms, peppercorn ranch dressing
       &
         Broiled top round steak, tomatoes, cucumber, baby yellow squash, zucchini, black olives, extra virgin olive oil\\\hline
  Meatloaf with tomato sauce, chopped sweet red bell peppers, extra virgin olive oil, cooked asparagus spears, sweet potatoes, orange, cantaloupe cubes
       &
         Chicken breast, breadcrumbs, fresh tomatoes, shredded green leaf
         lettuce, extra virgin olive oil, spinach, chopped yellow onion, sweet
         large yellow bell peppers, whole mushrooms, chili peppers, vinaigrette
  \\\hline
  Ciabatta bun, cooked skinless chicken breast, fresh baby spinach, shredded iceberg lettuce,
  shredded mozzarella cheese, ketchup, frozen yogurt bar
             &
               Small whole wheat submarine roll, broiled round roast
               beef, roasted light turkey meat without skin, fresh medium
               tomatoes, honey smoked ham, shredded iceberg lettuce, sliced mozzarella cheese\\
         \bottomrule
\end{tabular}
}
\caption[Qualitative evaluation of \textsc{rfs} for food recommendation]{\textbf{\acrlong{rfs} trained on food consumption data provides diverse meal recommendations.} \gls{rfs} with \Cref{eqn:residual} is fit to data from a diet tracking app; items are meals and attributes are the ingredients in the meal. Meals are represented the average of their attribute embeddings, and cosine similarity between meal representations is used to find the nearest neighbors of meals (user-level information cannot be shown as this is personal diet data). \gls{rfs} reveals eating patterns: for example, the second-last query meal is a mix of meat, vegetables, and fruit, and the nearest neighbor meal is a different meat with a side of salad; the last query meal is a sandwich, and its nearest neighbor is also a sandwich with different ingredients.}% This reveals
% that \acrlong{rfs} uncovers latent patterns of consumption in the attribute
% embeddings that can be leveraged to improve recommendations.
\label{tab:nearest_meals}
\end{table*}

To define scalable item embedding and item intercept functions, note that the parameterization of the item embedding function $g(x_m)$ depends on the size of the data. If the number of items is small, $g$ can function as a lookup for unique intercepts for every item. However, if the number of items is so large that unique item intercepts lead to overfitting, a scalable parameterization of item embeddings $g$ can be defined using additional information about every item. For example, if the data consists of foods in meals, we can define a meal intercept as the mean of food intercepts, yielding a scalable item intercept function. The item intercept function $h(x_m)$ that maps item attributes to scalars is constructed in the same way. We study both of these choices in \Cref{sec:rfs-experiments}.

The inner product regression function in \Cref{eqn:rankfromsets} has several benefits. It requires computing a sum over only the attributes with which each item is associated. This enables \gls{rfs} to scale to large attribute vocabularies where traditional matrix factorization methods are intractable. Second, the embed-and-average approach to set modeling is provably flexible as we show later. We now describe deep variants of \gls{rfs} and detail how \gls{rfs} can approximate other recommendation models.% Consider the example from \Cref{sec:rfs-introduction}
% of predicting whether a user
% would enjoy a meal. Each user has some preferences about which meals they enjoy.
% A meal is made up of foods or attributes. One does not need to consider foods
% that are not in a meal; unused foods can be ignored. In addition, a meal is more
% than its foods. For instance, a user trying to eat healthier may enjoy all the
% ingredients in a meal, but if it is deep fried, it will be unappealing. The
% regression function in \gls{rfs} should be able to learn such latent patterns in
% meal consumption. Finally, even if a meal is not very appealing to a user, it
% may be popular and available everywhere, making it more likely to be consumed.

% The regression function $f(u, x_m)$ parameterizes the generative process of the
% labels conditional on the item attributes and user,
% $\yum \sim \textrm{Bernoulli}\left(\yum; \sigma(f(u, x_m)\right)$.

% Specifically, recent results in deep learning for set-valued data show that
% such an approach is sufficient to model nearly any function
% \citep{DBLP:journals/corr/ZaheerKRPSS17}. In theory, it may be necessary to
% apply a nonlinearity, such as a deep neural network, to $f$ to learn any
% function. In preliminary experiments, we found increasing the embedding size
% $d$ in the embedding model offered a better tradeoff between parameter size
% and model performance than the additional parameters added by a neural
% network.

The \gls{rfs} inner product regression function in \Cref{eqn:rankfromsets} is a log-bilinear model. But there are several other choices of regression function, and we draw on the deep learning toolkit for classification to build two other example architectures. With finite data and finite compute, one architecture may outperform another, or prove insufficient to capture patterns in user consumption. (Later, we show that all architectures are equivalent under fewer assumptions.) First, as an alternative to the log-bilinear model in \Cref{eqn:rankfromsets}, we can use a deep neural network as a regression function:
\begin{align}
  f\left(u, x_m\right) = \phi\left(\theta_u, \frac{1}{|x_m|}\sum_{j\in x_m}
  \beta_j, g(x_m)\right) + h(x_m) \, ,
  \label{eqn:neural-network}
\end{align}
where the deep network $\phi$ has weights and biases and takes as inputs the user embedding, sum of attribute embeddings, and item intercept. Such a neural network can represent functions that may or may not include the inner product in \Cref{eqn:rankfromsets}; \emph{ex~ante}, it is unclear whether a finite-depth, finite-width neural network can represent the inner product.

Another regression function for \gls{rfs} is a combination of
\Cref{eqn:rankfromsets,eqn:neural-network}, using an idea borrowed from deep
residual networks for image classification~\citep{he2015deep}. In this
architecture, a neural network $\phi$ with the same inputs as in
\Cref{eqn:neural-network} learns the residual of the inner product model:
\begin{align}
  f\left(u, x_m\right) = \theta_u^\top\left(\frac{1}{|x_m|}\sum_{j\in x_m}
  \beta_j + g(x_m)\right) + \phi + h(x_m) \, .
  \label{eqn:residual}
\end{align}
The choice of regression function in \acrshort{rfs} depends on the data. On finite data, with finite compute, one parameterization of \gls{rfs} will outperform another. To demonstrate this, we simulated synthetic data from the same generative process \gls{rfs} employs with a ground-truth regression function (a square kernel), and found that the residual and deep parameterizations outperformed the inner product architecture. These results are included in \Cref{sec:simulation}, and motivate exploring other architectures than the three examples here.
% We conclude this section by showing that in the regime of infinite
% data and compute, all the \gls{rfs} architectures we propose, including the
% inner product, can approximate other recommendation models that operate on
% set-valued input such as matrix factorization.

Stepping back from the setting of finite data and compute, a bigger picture emerges, which reveals the choice of regression function in \gls{rfs} does not matter. We show that any \gls{rfs} architecture is sufficiently flexible to approximate recommendation models that operate on set-valued input. We define permutation-invariant models before deriving this result.

The regression function $f$ in \gls{rfs} operates on set-valued input: the unordered collection of item attributes $x_m$. A set is, by definition, permutation-invariant: it remains the same if we permute its elements. Functions that operate on set-valued inputs must also be permutation-invariant. \gls{rfs} is permutation-invariant; the set of attributes associated with an item enter into \Cref{eqn:rankfromsets,eqn:neural-network,eqn:residual} via summation. Other examples of permutation-invariant recommendation models are multiple matrix factorization, models based on word embeddings, and permutation-marginalized recurrent neural networks. These models are shown to be permutation-invariant in \Cref{sec:models} and evaluated in \Cref{sec:rfs-experiments}. We now show that \gls{rfs} can approximate other permutation-invariant recommendation models such as matrix factorization.

\begin{prop}
  Assume the vocabulary of attributes (set elements) is countable, $\lvert V \rvert < \lvert \bbN_0 \rvert$. Then \acrshort{rfs} can approximate any permutation-invariant recommendation model.
  \label{prop:universal-approximation}
\end{prop}
The proof follows directly from Theorem~2 in \citet{zaheer2017deep} and we will not
restate it here. (The only change to the proof is the mapping from set elements
to one-hot vectors, $c \colon V \to \left\{0, 1\right\}^{\lvert V \rvert}$ to
yield a unique representation of every object in the powerset.)
\Cref{prop:universal-approximation} means that any of the parameterizations in
\Cref{eqn:rankfromsets,eqn:neural-network,eqn:residual} is flexible enough to
approximate other principled recommendation models that leverage item
attributes, such as multiple matrix
factorization~\citep{gopalan2014content-based,wang2011collaborative}.% This proposition also supports
% exploring other parameterizations of \gls{rfs} that may have better
% computational or statistical properties.

The parameters for \gls{rfs} are learned by stochastic optimization. Denote
the full set of \gls{rfs} model parameters by $\mbgamma$, and let~$\cD_u$~be the
empirical data distribution for a user. Let $\lambda_u$ be a reweighting
parameter. The per-user maximum likelihood objective for \gls{rfs} is
\begin{equation}
  \label{eq:objective}
  \begin{aligned}
  \cL(\mbgamma, \lambda_u) = \E_u\big[ \E_{m \sim \cD_u \mid \yum = 1}%&
  \left
  [\log p(\yum =1 \mid x_m; \mbgamma)\right]
  + \lambda_u \E_{k \sim \cD_u \mid \yuk = 0}%&
  \left[\log p(\yuk = 0 \mid x_k;
  \mbgamma)\right]
                \big]
  \end{aligned}
\end{equation}

In traditional regression, altering the ratio of positive to negative examples
by reweighting leads to inconsistent parameter estimation. The inconsistency
stems from the randomness in the labels, given the features. However, Recall@$K$ assumes that each user, item attribute set pair ($u, x_m$) uniquely determines
whether the item was consumed or not (the label $\yum$). Here, all reweightings
produce the same result. This means that for any negative example weight
$\lambda_u$, the learned model will be the same. In practice we set $\lambda_u$
to balance the positive and negative examples for each user. We use stochastic
optimization to maximize \Cref{eq:objective}, and describe two negative sampling
schemes that are dependent on the choice of evaluation metric.
% for a single datapoint $(x_m, \yum=1)$ is
% \begin{align}
% \begin{split}
%   \cL(\theta, & (x_m, \yum), \{(x_k, \yuk)\}) =\\
%   &\log p(\yum = 1 \mid x_m; \theta) + \sum_{k=1}^L \log p(\yuk = 0 \mid x_k;
%   \theta),\\
%   &\textrm{with}~x_k\sim\textrm{Uniform}(I).
%   \label{eq:objective}
% \end{split}
% \end{align}
% As users only interact with a small number of items, this objective function is
% dominated by negatively-labeled datapoints with~$\yum = 0$. We use stochastic
% optimization to maximize \Cref{eq:objective} and learn parameters jointly.
% Stochastic optimization requires subsampling the terms in the objective with
% negative labels. Such subsampling directly leads to a binary classification
% binary classification loss function with negative samples, as in previous work
% on word embeddings~\citep{mikolov2013distributed} and recommender
% systems~\citep{he2017neural,song2018neural}.
% The \gls{rfs} parameterization in
% \Cref{eqn:prob_y_given_um,eqn:argmax,eqn:rankfromsets} is bi-convex and could be
% fit through alternating coordinate descent. That is, fixing the user embeddings,
% the problem is convex in each of the other parameters; similarly, the problem is
% convex in the user parameters if the other parameters are fixed. For each
% coordinate in the user step, such a procedure would require iterating over all
% items. The massive number of items in the datasets in \Cref{sec:rfs-experiments}
% makes this intractable; similar issues could also arise for datasets with many
% users.
% \paragraph{Negative sampling} All observed positive labels (i.e. instances where
% $\yum = 1$) are divided into mini-batches. For each mini-batch, each
% positively-labeled sample is paired with a negatively-labeled sample. The
% negatively-labeled sample has the same user as its paired sample, but the item
% is sampled uniformly from a set of items. This procedure is not unbiased, but
% still leads to a consistent classifier: note that any reweighting of items that
% assigns nonzero weight to all items can be arbitrarily biased, but still lead to
% a consistent classifier. As long as every item has nonzero weight, the infinite
% data limit ensures that the classifier will learn to discriminate between
% positively- and negatively-labeled datapoints. We decide to balance
% positively-labeled datapoints with negative samples in each mini-batch as it is
% easy to implement as shown in \Cref{sec:code}.
% We note that any negative sampling
% scheme that assigns nonzero weight to every item leads to a consistent
% classifier.

% Ideally, the sampling set should be the set of all items which the user has not
% consumed, where~$\yum~=~0$. But maintaining and sampling from user-specific sets
% of negative labels can be computationally prohibitive. When the item set is
% large and labels are sparse (that is, $\yum = 0$ for most user, item pairs),
% embedding models are often trained by drawing negative samples uniformly from
% the item set~\citep{mikolov2013distributed}. For training \gls{rfs}, we propose
% two different negative sampling distributions that trade off quality of the
% learned parameters for scalability of the training procedure.

Negative samples can be drawn uniformly over the entire corpus of items, which we define to be corpus sampling. If the item set is large, this can be an expensive procedure. This negative sampling scheme leads to objective functions used in other recommender systems~\citep{he2017neural,song2018neural}.

On large datasets, it is infeasible to calculate Recall@$K$ for evaluation, as this requires ranking every item for every user (e.g. in \Cref{sec:rfs-experiments} we study a dataset with over $10$M items). We define a scalable evaluation metric based on recall, and describe how it leads to a natural choice of negative sampling distribution.

Sampled recall is defined as follows. Consider held-out datapoints with positive labels,~${(x_m, \yum = 1)}$. For every held-out datapoint, $K-1$ datapoints with negative labels~${(x_k, \yuk = 0)}$ are sampled from the rest of the held-out data, which together yield a set of $K$ datapoints. A recommendation model is used to rank the $K$ datapoints $r_{u1}, \ldots, r_{uK}$. SampledRecall@$k$ is the fraction of the $K$ held-out datapoints that the model ranks in the top~$k$:
\begin{equation}
  \textrm{SampledRecall@}k = \frac{1}{K}{\E_{um}\left[\sum_{r\in
\{\mb{r}_{u1},
\ldots \mb{r}_{uk}\}} y_{ur}\right]}\, .
\label{eq:sampled-recall}
\end{equation}

The expectation is over users and items in the held-out set of datapoints. This evaluation metric is scalable: instead of using a model to rank every item, SampledRecall@$k$ requires ranking only $K$ items. Sampled recall is~$1$~if $k~=~K$, as the held-out datapoint with $\yum = 1$ is in each list of $K$ datapoints to be ranked. This metric is used in recommender systems when the number of items is large~\citep{ebesu2018collaborative,yang2018openrec:}.
% For
% example, with $M=10$, an average sampled recall@1 of $0.5$ means for 50\% of the
% held-out datapoints, the item consumed by the user was ranked first by the model
% (out of $9$~items randomly sampled from the held-out data of other users). For
% the sampled recall metric we choose $M=10$, meaning $9$ `fake' meals are sampled
% uniformly from other users' meals in the held-out validation and test sets.

When sampled recall is used as an evaluation metric, batch sampling is a natural way to draw negative samples. Sampled recall is calculated on items drawn from other user's data. We define batch sampling as generating negative samples by permuting mini-batch items. Besides corresponding to the sampled recall metric, this technique is memory-efficient, as it requires that only the current mini-batch be in memory.

In addition to scalability, both negative sampling procedures above have
the advantage of implicitly balancing the classifier. As shown in
\citet{veitch2019empirical}, using stochastic gradient descent with
negative sampling is equivalent to a Monte Carlo approximation of the
reweighted (balanced) classification loss.
